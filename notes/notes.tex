\documentclass[11pt]{article}
% Basic packages
\usepackage[utf8]{inputenc}
\usepackage[italian]{babel}
\usepackage{hyperref}
% Mathematical packages
\usepackage{amsmath,amssymb,amsthm}
% Layout and formatting packages
\usepackage{geometry}
\usepackage{multicol}
\usepackage{fancyhdr}
\usepackage{titlesec}
% Color and box packages
\usepackage{xcolor}
\usepackage{tcolorbox}
\tcbuselibrary{theorems,skins,breakable}
% Page settings
\geometry{margin=2.5cm}
% Header settings
\pagestyle{fancy}
\fancyhf{}
\fancyhead[L]{Constraint Programming}
\fancyhead[R]{\thepage}
% Title customization
\titleformat{\section}{\Large\bfseries}{\thesection}{1em}{}
\titleformat{\subsection}{\large\bfseries}{\thesubsection}{1em}{}

% Colored environment definitions (without automatic numbering)
\newtcolorbox{theorem}[2][]{
enhanced,
colback=blue!5,
colframe=blue!70!black,
fonttitle=\bfseries,
rounded corners,
title={Teorema: #2},
#1
}

\newtcolorbox{definition}[2][]{
enhanced,
colback=teal!5,
colframe=teal!70!black,
fonttitle=\bfseries,
rounded corners,
title={Definizione: #2},
#1
}

\newtcolorbox{lemma}[2][]{
enhanced,
colback=gray!10,
colframe=gray!70!black,
fonttitle=\bfseries,
rounded corners,
title={Lemma: #2},
#1
}

\newtcolorbox{example}[2][]{
enhanced,
colback=green!8,
colframe=green!60!black,
fonttitle=\bfseries,
rounded corners,
title={Esempio: #2},
#1
}

\newtcolorbox{question}[2][]{
enhanced,
colback=red!5,
colframe=red!70!black,
fonttitle=\bfseries,
rounded corners,
title={Domanda: #2},
#1
}

\newtcolorbox{note}[2][]{
enhanced,
colback=violet!8,
colframe=violet!60!black,
fonttitle=\bfseries,
rounded corners,
title={Nota: #2},
#1
}

\newtcolorbox{problem}[2][]{
enhanced,
colback=orange!8,
colframe=orange!70!black,
fonttitle=\bfseries,
rounded corners,
title={Problema: #2},
segmentation style={dashed,orange!70!black,line width=1pt},
#1
}

% Custom commands
%\newcommand{\calU}{\mathcal{U}}
% Spacing settings
\setlength{\parindent}{0pt}      % No indentation
\setlength{\parskip}{1em}        % Space between paragraphs

% Usage examples:
% 
% \begin{theorem}{Nome del teorema}
% Testo del teorema
% \end{theorem}
% 
% \begin{definition}{Nome della definizione}
% Testo della definizione
% \end{definition}
% 
% \begin{lemma}{Nome del lemma}
% Testo del lemma
% \end{lemma}
%
% \begin{example}{Nome dell'esempio}
% Testo dell'esempio
% \end{example}
%
% \begin{question}{Nome della domanda}
% Testo della domanda
% \end{question}
%
% \begin{note}{Titolo della nota}
% Testo della nota
% \end{note}
%
% \begin{problem}{Nome del problema}
% Testo del problema da risolvere
% \tcblower
% \textbf{Soluzione:} Spiegazione della soluzione
% \end{problem}
%

\begin{document}

% Titolo
\begin{center}
    \LARGE \textbf{Constraint Programming}
\end{center}

\vspace{0.5cm}
\hrule
\vspace{0.5cm}

\tableofcontents

\vspace{0.5cm}
\hrule
\vspace{0.5cm}

\section{Lezione 1 - 18/09/2023}
\subsection{ACM-IEEE: IS - Intelligent Systems}
    AI is the study of solutions for problems that are difficult or impractical to solve with traditional methods. The solutions rely on a broad set of:
    \begin{itemize}
        \item general and specialized knowledge representation schemes
        \item problem solving mechanisms
        \item learning techniques
    \end{itemize}

    12 subareas:
    $$\begin{array}{ll}
        \color{orange}\text{IS/Fundamental ISSUES}  &  \text{IS/Basic Machine Learning} \\
        \color{orange}\text{IS/Basic Knowledge Representation and Reasoning} & \text{IS/Advanced Machine Learning} \\
        \color{orange}\text{IS/Basic Search Strategies} & \text{IS/Reasoning Under Uncertainty} \\
        \color{orange}\text{IS/Advanced Search} & \text{IS/Natural Language Processing} \\
        \text{IS/Advanced Representation and Reasoning} & \text{IS/Robotics} \\
        \text{IS/Agents} & \text{IS/Perception and Computer Vision}
    \end{array}$$

    \textcolor{orange}{IS/Fundamental ISSUES}
    \begin{itemize}
        \item Overview of AI problems, examples of successful recent AI applications
    \end{itemize}
    
    \textcolor{orange}{IS/Basic Knowledge Representation and Reasoning}
    \begin{itemize}
        \item Review of propositional and predicate logic
        \item Resolution and theorem proving
    \end{itemize}

    \textcolor{orange}{IS/Basic Search Strategies}
    \begin{itemize}
        \item Problem spaces (states, goals and operators), problem solving by search
        \item Factored representation (factoring state into variables)
        \item Uninoformed search (breadth-first, depth-first, depth-first with iterative deepening)
        \item Heuristics and informed search (hill-climbing, generic best-first, A*)
        \item Space and time efficiecy of search
        \item Constraint satisfaction (backtracking and local search methods)
    \end{itemize}

    \textcolor{orange}{IS/Advanced Search}
    \begin{itemize}
        \item Global constraints
        \item Large Neighborhood Search
        \item (Parallelism)
    \end{itemize}

    See conferences: \href{iclp25.demacs.unical.it}{ICLP}, \href{cp2025.a4cp.org}{CP}, \href{2025.ijcai.org}{IJCAI}

\subsection{Introduction}
    CP is a declarative programming paradigm suited for modeling and solving complex problems. Problem modeling and solution searching are clearly separated and typically the code is very readable and easy to modify.
    
    You don't have restrictions on the kind of constraints.
    Solution search is natural to parallelize and search heuristics are crucial.

\subsection{The future of xAI}
    Thanks to the growth of computing resources, the learning capability of artificial intelligence has made the subsymbolic approach very popular. On a larger scale, an exponential phenomenon is complex to handle.

    Furthermore, one of the current concerns is the amount of energy required to train the models.

    \subsubsection{EU AI ACT}
        In 2024, the EU published the world's first law comprehensively regulating AI. They have developed a set of rules for scenarios that serve as guidelines so that people can develop and use AI.

        \begin{enumerate}
            \item AI definition
            \item Forbidden activities: social scoring, remote biometrics, subliminal methods / fragile people
            \item Risk based classification of AI:
                \begin{itemize}
                    \item Medical and legal applications and scoring (students, financial, CV) are High Risk
                \end{itemize}
        \end{enumerate}

        GenAI is not explicitly included in AI Act.
        
        New issues:
        \begin{itemize}
            \item Copyright
            \item Bulk data retrieval
            \item Legal liability for generated output
            \item Embedding into other services (API)
        \end{itemize}

    \subsubsection{GenAI Risks}
        \textbf{Bias}: AI learns from unbalanced datasets $\rightarrow$ this is our society $\rightarrow$ discriminations

        \textbf{Blackbox}:
        \begin{itemize}
            \item Algorithms / training dataset not available
            \item Can't analyze the trained network (propietary)
            \item Even if available $\rightarrow$ trilions of parameters
        \end{itemize}

        \textbf{Improper usage}: AI for a specific and unruled task (e.g. fake news)

    \subsubsection{Explainable AI}
        Ethical principles to include AI in the decisional chain:
        \begin{itemize}
            \item System's trust
            \item Compliance to standards/directives
            \item Bias control
            \item Incremental improving
        \end{itemize}

        How?
        \begin{itemize}
            \item Explainability allows to understand how/why I get that answer
            \item Transparency and interpretability (access to algorithms and data)
            \item Logical proof
        \end{itemize}

        Creating new AI to explain AI is like creating blackbox to explain blackbox. A solution could be Native Explainability.

    \subsubsection{Two joining paths}
        $\begin{array}{ll}
            \textbf{ML}   &   \textbf{Symbolic AI} \\
            \text{Learns easily}    &  \text{Knowledge representation} \\
            \text{No programming}   &   \text{Programming (e.g. constraints)} \\
            \text{Blackbox} &   \text{Natively explainable} \\
            \text{Expensive training}   &   \text{Expensive computation} \\
            \text{Syntax level} &   \text{Semantic level} \\
            \text{Probability for next token}   &   \text{Relations on objects} \\
            \text{NL expressions patterns}  &   \text{Similar to human deduction} \\
            \text{Somehow captures a certain semantica} &   \text{Can explain ML?}
        \end{array}$
        
\vspace{0.5cm}
\hrule
\vspace{0.5cm}

\section{Lezione 2 - 20/09/2023}

\end{document}